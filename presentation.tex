%%%%%%%%%%%%%%%%%%%%%%%%%%%%%%%%%%%%% 
%% LE2I beamer template
%% Guillaume Lemaitre, October 2014
%%%%%%%%%%%%%%%%%%%%%%%%%%%%%%%%%%%%% 

\documentclass{beamer}

\usepackage{listings}
\usepackage[utf8]{inputenc}
\usepackage[T1]{fontenc} 
\usetheme{le2i} 

%% The amssymb package provides various useful mathematical symbols
\usepackage{amssymb}
%% The amsthm package provides extended theorem environments
\usepackage{amsthm}
%% amsmath for math environment
\usepackage{amsmath}

\DeclareMathOperator*{\argmin}{arg\,min}
\DeclareMathOperator*{\argmax}{arg\,max}
\DeclareMathOperator*{\sign}{sign}

%% figure package
\usepackage{epsf,graphicx}
\usepackage{epstopdf}
\usepackage{subfigure}
\usepackage{transparent}

%% In order to draw some graphs
\usepackage{tikz,xifthen}
\usepackage{tikz-qtree}
\usepackage{adjustbox}
\usetikzlibrary{decorations.pathmorphing}
\usetikzlibrary{fit}
\usetikzlibrary{backgrounds}
\usetikzlibrary{shapes,arrows,shadows}
\usetikzlibrary{calc,decorations.pathreplacing,decorations.markings,positioning}
\usetikzlibrary{snakes,decorations.text,shapes,patterns}
% \usepackage{scalefnt,lmodern,booktabs}

%% Package for cross and tick symbols
\usepackage{pifont}
\newcommand{\tick}{\color{green!60!black!80}\ding{51}}
\newcommand{\cross}{\color{red!60!black!80}\ding{55}}

\title{Computer Science}
\author{Cédric Lemaître \\ \texttt{c.lemaitre58@gmail.com}}
\date{BScv 2017 - 2018 \\ 2\textsuperscript{nd} semester}

\institute{Universit\'e de Bourgogne} 

%% Uncomment if you want to avoid thousand of bullet inside the menu
% \usepackage{etoolbox}
% \makeatletter
% \patchcmd{\slideentry}{\advance\beamer@xpos by1\relax}{}{}{}
% \def\beamer@subsectionentry#1#2#3#4#5{\advance\beamer@xpos by1\relax}%
% \makeatother

\begin{document}

% Show the title page
\begin{frame}
  \titlepage
\end{frame}

% Show the table of contents
\begin{frame}
  \tableofcontents[sectionstyle=show,subsectionstyle=show,subsubsectionstyle=hide]
\end{frame}

\section{Computer Science : C++}

\subsection{C++ Language : introduction}

\begin{frame}
  \frametitle{Objectives of the module}
  \begin{block}{Objectives}
    \begin{itemize}
	\item learn C++ language
	\item learn Object Programmation Concept
	\item learn Optimized compilation chain
    \end{itemize}
  \end{block}
\end{frame}


\begin{frame}
  \frametitle{C++ introduction}
  \begin{block}{Chronology}
    \begin{itemize}
	\item first version 1983 by Bjarne Stroustrup
	\item first standardized version 1998
	\item last standardized version 2017
    \end{itemize}
  \end{block}
\end{frame}

\begin{frame}
  \frametitle{C++ introduction}
  \begin{block}{What is C++?}
    \begin{itemize}
	\item most used language for software dev
	\item object oriented language
    \end{itemize}
  \end{block}
\end{frame}

\begin{frame}
  \frametitle{C++ introduction}
  \begin{block}{Advantages C++}
    \begin{itemize}
		\item high perf (vs python)
		\item cross-plateform
		\item object representation : increase dev flow
		\item many object in standard lib
    \end{itemize}
  \end{block}
\end{frame}

\subsection{Comments}
\label{sub:comments}

\begin{frame}[fragile]
  \frametitle{Comments}
\begin{lstlisting}[language=C++,keywordstyle=\color{red}]
  // this is a inline comment
  CStash intStash; // you could comment like this
  /* this is block comment
  int i;
  char* cp;
  ifstream in;
  */
\end{lstlisting}
\end{frame}
\subsection{Type}
\label{sub:type}


\begin{frame}
  \frametitle{Standard type}
  \begin{block}{integer}
    \begin{itemize}
		\item char
		\item short
		\item int
		\item long
    \end{itemize}
	Each one could be unsigned or not.
  \end{block}
\end{frame}

\begin{frame}
  \frametitle{Standard type}
  \begin{block}{real}
    \begin{itemize}
		\item float
		\item double
		\item long double
    \end{itemize}
	Each one could be unsigned or not.
  \end{block}
\end{frame}

\begin{frame}
  \frametitle{Standard type}
  \begin{block}{integer}
	  Be carefull : exact size of type variables depend of the compiler manufacturer  
  \end{block}
\end{frame}

\subsection{variable definition}
\label{sub:variable_definition}

\begin{frame}
  \frametitle{variable definition}
	\begin{itemize}
		\item definition
		\item affection and assignment
	\end{itemize}
\end{frame}

\subsection{instruction}
\label{sub:instruction}

\begin{frame}
  \frametitle{instruction}
	\begin{itemize}
		\item all instruction finish with ;
	\end{itemize}
\end{frame}

\begin{frame}
  \frametitle{operator}
	\begin{itemize}
		\item maths
		\item logical
	\end{itemize}
\end{frame}

\subsection{fonction}
\label{sub:fonction}

\begin{frame}
  \frametitle{function}
	\begin{itemize}
		\item Definition
		\item Call
	\end{itemize}
\end{frame}


\begin{frame}
  \frametitle{function}
	\begin{itemize}
		\item where place the definition?
		\item overloading a function
	\end{itemize}
\end{frame}

\subsection{Our first binary}
\label{sub:first_programm}

\begin{frame}
  \frametitle{Our first binary}
	\begin{itemize}
		\item code structure
		\item compilation : g++  -Wall  -o labs1.o   -c labs1.cpp
		\item linking : g++ -o labs1  labs1.o
		\item run binary ./labs1
	\end{itemize}
\end{frame}

\subsection{Pointer and Dynamic Memory Allocation}
\label{sub:pointer_and_dynamic_memory_allocation}

\begin{frame}
  \frametitle{Address and Value}
	\begin{itemize}
		\item How to define a memory address of variable?
		\item How to change a value to a memory address?
		\item How to get the address of a variable?
	\end{itemize}
\end{frame}

\begin{frame}
  \frametitle{How to allocate a vector}
	\begin{itemize}
		\item Static case : can't change the size
		\item Dynamic Allocation
		\item Free memory
	\end{itemize}
\end{frame}

\subsection{Function (2)}
\label{sub:function_2_}

\begin{frame}
	\frametitle{function (2)}
	\begin{itemize}
		\item argument with reference
		\item argument with address
		\item const argument
		\item inline fonction
	\end{itemize}
\end{frame}

\subsection{Conditionnal structure and loop}
\label{sub:condition}

\begin{frame}
	\frametitle{Conditionnal structure and loop}
	\begin{itemize}
		\item if else
		\item switch case
		\item while
		\item do while
		\item for
	\end{itemize}
\end{frame}

\subsection{Class}
\label{sub:class}

\begin{frame}
	\frametitle{Class}
	\begin{itemize}
		\item member functions/methods
		\item member variables
		\item private
		\item protected
		\item public
	\end{itemize}
\end{frame}

\begin{frame}
	\frametitle{Class}
	\begin{itemize}
		\item constructors (Could be overloaded) 
		\item destructor (Unique)
		%\item friend class
		\item inheritance
		\item operator overload
	\end{itemize}
\end{frame}

\subsection{Link}
\label{sub:link}

\begin{frame}
	\frametitle{Link and document}
	\begin{itemize}
		\item Google Guide Style : some great practices \url{https://google.github.io/styleguide/cppguide.html}
		\item C++ resources network : reference help : \url{http://www.cplusplus.com/}
		\item VIM : IDE for dev : \url{http://www.vim.org/} 
		\item Emacs : IDE for dev : \url{https://www.gnu.org/software/emacs/} 
		\item QT Creator : IDE for dev \url{http://doc.qt.io/qtcreator/index.html} 
	\end{itemize}
\end{frame}

\subsection{Books}
\label{sub:books_}


\begin{frame}
	\frametitle{Books}
	\begin{itemize}
		\item B. Stroustrup: Programming -- Principles and Practice Using C++ (Second Edition)
		\item Bjarne Stroustrup: The C++ Programming Language (4th Edition)
	\end{itemize}
\end{frame}



\end{document}
